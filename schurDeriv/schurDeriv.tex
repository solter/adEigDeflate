\documentclass[fleqn,12pt]{article}

\usepackage{amsmath}
\usepackage{amssymb}
\usepackage[margin=1in]{geometry}

\newcommand{\bmatTt}[2]{\begin{bmatrix} #1 \\ #2 \end{bmatrix}}
\newcommand{\bmatTw}[2]{\begin{bmatrix} #1 & #2 \end{bmatrix}}
\newcommand{\bmatF}[4]{\begin{bmatrix} #1 & #2\\ #3 & #4 \end{bmatrix}}

\begin{document}
\section{The Derivative of a Schur Decomposition}
\subsection{Definitions and Properties}

For an arbitrary matrix $B$, let
\begin{equation}\label{eq:taylor}
  B(\alpha) = B + \alpha \dfrac{\partial B}{\partial \alpha} =: B + \alpha \dot{B}
\end{equation}
and let $B^*$ represent the hermitian transpose of $B$ (\textit{matlab}'s $B'$).

A Schur Decomposition is defined as
\begin{equation}\label{eq:Schur}
  Q^* A Q = S \Leftrightarrow A Q = Q S
\end{equation}
Where $Q$ is orthogonal, and $S$ is block triangular.

The last piece we need is to define $U$ as follows\footnote{
Why we can make the assumption that $U$ takes this form is beyond me.
But Anderson does...
}
\begin{align}
  Q(\alpha) &= Q + \alpha \dot{Q} \\
  &= Q (I + \alpha \tilde{U}) \\
  &= Q(0) \cdot U(\alpha)\\
  U &= U^{(1)} U^{(2)}\\
  U^{(1)} &= \begin{bmatrix}
    I_k & - P^*\\
    P & I_{n-k}
  \end{bmatrix}\\
  U^{(2)} &= \begin{bmatrix}
    (I_{k} + P^* P)^{-1/2} & 0\\
    0 & (I_{n - k} - P P^*)
  \end{bmatrix}
\end{align}

Note that as $\alpha \to 0$, $U \to I$, leading to the following as $\alpha \to 0$
\begin{gather}
  P \to 0\\
  \dfrac{\partial}{\partial \alpha} (I_k + P^* P)^{-1/2} \to 0\\
  \dfrac{\partial}{\partial \alpha} (I_{n-k} + P P^*)^{-1/2} \to 0\\
  U(0) = I\\
  \left. \dfrac{\partial U}{\partial \alpha} \right|_{\alpha = 0} = \begin{bmatrix}
    I_k & - \dot{P}^*\\
    \dot{P} & I_{n-k}
  \end{bmatrix}
\end{gather}

Thus
\begin{equation}\label{eq:QbyP}
  \dot{Q} = Q \cdot \dot{U}(0) = Q \cdot \begin{bmatrix}
    I_k & - \dot{P}^*\\
    \dot{P} & I_{n-k}
  \end{bmatrix}
\end{equation}
and finding $\dot{A}$ is now reduced to finding $\dot{P}$, which can be done as follows.

\subsection{Finding $\dot{P}$}

Using definition \ref{eq:taylor} in equation \ref{eq:Schur}, differentiating, and evaluating at $\alpha = 0$ gives
\begin{gather}
  \dfrac{\partial}{\partial \alpha} 
  \left[ 
    (A + \alpha \dot{A}) Q U(\alpha) = Q U(\alpha) S(\alpha)  
  \right]_{\alpha = 0}\\
  \dot{A} Q U(0) + A Q \dot{U}(0) = Q \dot{U}(0) S + Q U(0) \dot{S}
\end{gather}
Where $A = A(0)$, $Q = Q(0)$, and $S = S(0)$.
Which left multiplying by $Q^*$ and using $U(0) = I$ reduces to
\begin{equation}
  Q^* \dot{A} Q + S \dot{U}(0) = \dot{U}(0) S + \dot{S}\\
\end{equation}
Representing $Q$ into $[ Q_1 Q_2]$, where $Q_1 \in \mathbb{R}^{n \times (n-k)}$,
and breaking $S$ into 4 components such that $S_{1,1} \in \mathbb{R}^{k \times k}$ leads
to the previous statement being represented by
\begin{equation}
  \bmatTt{Q_1^*}{Q_2^*} \dot{A} \bmatTw{Q_1}{Q_2} + 
  \bmatF{S_{1,1}}{S_{1,2}}{0}{S_{2,2}} \bmatF{I_k}{-\dot{P}^*}{\dot{P}}{I_{n-k}} = 
  \bmatF{\dot{S}_{1,1}}{\dot{S}_{1,2}}{0}{\dot{S}_{2,2}} + 
  \bmatF{I_k}{-\dot{P}^*}{\dot{P}}{I_{n-k}} \bmatF{S_{1,1}}{S_{1,2}}{0}{S_{2,2}}
\end{equation}
Examining just the lower left corner of the previous equation, namely rows $n-k : n$ and columns $k:n$
leads to the following subproblem:
\begin{equation}
  Q_2^* \dot{A} Q_1 + S_{2,2} \dot{P} = \dot{P} S_{1,1}
\end{equation}
Which reorganized is
\begin{equation}
  -S_{2,2} \dot{P} + \dot{P} S_{1,1} = Q_2^* \dot{A} Q_1
\end{equation}
If $k$ is chosen such that $k = n - k$, then this is the Sylvester Equation for which exist LAPACK routines.
Better yet, Octave itself can solve this for $\dot{P}$ using the command
\begin{equation}
  \dot{P} = syl \left( -S_{2,2} , \; S_{1,1} , \; Q_2^* \dot{A} Q_1 \right)
\end{equation}
Using equation \ref{eq:QbyP} now yields the desired $\dot{Q}$ needed for our algorithm!

\end{document}
